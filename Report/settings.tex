%% Cleveref
\crefname{listing}{Code}{Codes} % Match with minted caption

%% Pgfplots
\pgfplotsset{compat=newest,scaled ticks = false,
            tick label style={/pgf/number format/fixed},
            /pgf/number format/1000 sep={}}

\usetikzlibrary{spy}
%\usetikzlibrary{external}
%\tikzexternalize
%\tikzsetexternalprefix{tikz/}

%% Tocloft
% Set the name of figures and tables in the list of figures and tables
\newlength{\fignamelength}
\settowidth{\fignamelength}{\tablename}
\addtolength{\cftfignumwidth}{\fignamelength}
\renewcommand{\cftfigpresnum}{\figurename~}

\newlength{\tabnamelength}
\settowidth{\tabnamelength}{\figurename}
\addtolength{\cfttabnumwidth}{\tabnamelength}
\renewcommand{\cfttabpresnum}{\tablename~}

%% Siunitx
\sisetup{exponent-product=\cdot, % Use \cdot instead of \times as replacement for 10-exponentials
        output-complex-root=\ensuremath{i}, % Normal math-i instead of \mathrm{i},
        group-separator = {}} % Don't separate thousands

%% Biblatex
\iftoggle{biblatexloaded}{
    % Surname first
    \DeclareNameAlias{sortname}{last-first}
    \DeclareNameAlias{default}{last-first}

    \urlstyle{sf}
}

%% Minted
\iftoggle{mintedloaded}{
    % C
    \newmintedfile{c}{fontfamily=tt,
                        fontsize=\footnotesize,
                        tabsize=4,
                        numberblanklines=true,
                        numbers=left,
                        numbersep=5pt,
                        breakautoindent=false,
                        xleftmargin=0.7cm,} % \cfile{}

    \newminted[ccodecap]{c}{fontfamily=tt,
                            fontsize=\normalsize,
                            tabsize=4,
                            frame=lines,
                            breaklines=true,
                            breaksymbolleft=\tiny\ensuremath{\hookrightarrow},
                            breakautoindent=true,} %\begin{ccodecap}
    \newmintinline{c}{fontfamily=tt,
                        breaklines=true,
                        fontsize=\normalsize} %\cinline{}

    % Bash
    \newminted[bashcode]{bash}{fontfamily=tt,
                                            fontsize=\normalsize,
                                            tabsize=4,} %\begin{bashcode}

    \newminted[bashcodecap]{bash}{fontfamily=tt,
                                            fontsize=\normalsize,
                                            tabsize=4,
                                            frame=lines,
                                            breaklines=true,
                                            breaksymbolleft=\tiny\ensuremath{\hookrightarrow},
                                            breakautoindent=true} %\begin{bashcodecap}
    \newmint[bash]{bash}{fontfamily=tt,
                                    fontsize=\normalsize,
                                    frame=lines,
                                    breaklines=true,
                                    breaksymbolleft=\tiny\ensuremath{\hookrightarrow},
                                    breakautoindent=true} % \bash{}
    \newmintedfile[bashfile]{bash}{fontfamily=tt,
                                                fontsize=\normalsize,
                                                tabsize=4,
                                                breakautoindent=false,} % \bashfile{}
    \newmintinline[bashinline]{bash}{fontfamily=tt,
                                                fontsize=\normalsize,
                                                breaklines=true,
                                                breakautoindent=false} % \bashinline{}
    % MATLAB
    \newmintedfile{matlab}{fontfamily=tt,
                            fontsize=\footnotesize,
                            tabsize=4,
                            numberblanklines=true,
                            numbers=left,
                            numbersep=5pt,
                            breakautoindent=false,
                            xleftmargin=0.7cm,} % \matlabfile{}
    \newmintedfile[matlabfileframe]{matlab}{fontfamily=tt,
                                            fontsize=\footnotesize,
                                            tabsize=4,
                                            frame=lines,
                                            numberblanklines=true,
                                            numbers=left,
                                            numbersep=5pt,
                                            breakautoindent=false,
                                            xleftmargin=0.7cm,} % \matlabfileframe{}


    \newminted[mcodecap]{matlab}{fontfamily=tt,
                                    fontsize=\normalsize,
                                    tabsize=4,
                                    frame=lines,
                                    breaklines=true,
                                    breaksymbolleft=\tiny\ensuremath{\hookrightarrow},
                                    breakautoindent=true,} %\begin{mcodecap}

    \newminted[mcode]{matlab}{fontfamily=tt,
                                    tabsize=4,
                                    breaklines=true,
                                    breaksymbolleft=\tiny\ensuremath{\hookrightarrow},
                                    breakautoindent=true,} %\begin{mcode}

    \newmintinline[minline]{matlab}{fontfamily=tt,
                                    breaklines=true,
                                    fontsize=\normalsize} %\minline{}

    \SetupFloatingEnvironment{listing}{name=Code} % Shows "Code" in captions
}

%% Misc
% Captions of floats
\usepackage[margin=3ex,
			font=small,
			labelfont=bf,
			labelsep=endash]{caption}

% Links
\hypersetup{colorlinks=true,
			hidelinks}
			
			
% Definitions of commands
\newcommand{\mail}[1]{\href{mailto:#1}{\nolinkurl{#1}}} % Email addresses as links
\newcommand{\pd}[0]{\partial} % Partial differential d
\newcommand{\matr}[1]{#1} % Matrices
\newcommand{\trps}[0]{\mathrm{T}} % Transpose of vector or matrices
\DeclareMathOperator{\linspan}{span} % Span of spaces in linear algebra
\newtheorem{theorem}{Theorem} % \begin{theorem}[Name of Theorem]
\renewcommand{\exp}[1]{\mathrm{e}^{#1}} % Exponential e

% Redefine \left and \right in order to make them look nicer when used in functions
\let\originalleft\left
\let\originalright\right
\renewcommand{\left}{\mathopen{}\mathclose\bgroup\originalleft}
\renewcommand{\right}{\aftergroup\egroup\originalright}

% Folder with figures
\graphicspath{./figs/}
